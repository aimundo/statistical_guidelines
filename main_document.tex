% !TeX program = pdfLaTeX
\documentclass[smallextended]{svjour3}       % onecolumn (second format)
%\documentclass[twocolumn]{svjour3}          % twocolumn
%
\smartqed  % flush right qed marks, e.g. at end of proof
%
\usepackage{amsmath}
\usepackage{graphicx}
\usepackage[utf8]{inputenc}

\usepackage[hyphens]{url} % not crucial - just used below for the URL
\usepackage{hyperref}

%
% \usepackage{mathptmx}      % use Times fonts if available on your TeX system
%
% insert here the call for the packages your document requires
%\usepackage{latexsym}
% etc.
%
% please place your own definitions here and don't use \def but
% \newcommand{}{}
%
% Insert the name of "your journal" with
% \journalname{myjournal}
%

%% load any required packages here



% tightlist command for lists without linebreak
\providecommand{\tightlist}{%
  \setlength{\itemsep}{0pt}\setlength{\parskip}{0pt}}

% From pandoc table feature
\usepackage{longtable,booktabs,array}
\usepackage{calc} % for calculating minipage widths
% Correct order of tables after \paragraph or \subparagraph
\usepackage{etoolbox}
\makeatletter
\patchcmd\longtable{\par}{\if@noskipsec\mbox{}\fi\par}{}{}
\makeatother
% Allow footnotes in longtable head/foot
\IfFileExists{footnotehyper.sty}{\usepackage{footnotehyper}}{\usepackage{footnote}}
\makesavenoteenv{longtable}

% Pandoc citation processing
\newlength{\cslhangindent}
\setlength{\cslhangindent}{1.5em}
\newlength{\csllabelwidth}
\setlength{\csllabelwidth}{3em}
\newlength{\cslentryspacingunit} % times entry-spacing
\setlength{\cslentryspacingunit}{\parskip}
% for Pandoc 2.8 to 2.10.1
\newenvironment{cslreferences}%
  {}%
  {\par}
% For Pandoc 2.11+
\newenvironment{CSLReferences}[2] % #1 hanging-ident, #2 entry spacing
 {% don't indent paragraphs
  \setlength{\parindent}{0pt}
  % turn on hanging indent if param 1 is 1
  \ifodd #1
  \let\oldpar\par
  \def\par{\hangindent=\cslhangindent\oldpar}
  \fi
  % set entry spacing
  \setlength{\parskip}{#2\cslentryspacingunit}
 }%
 {}
\usepackage{calc}
\newcommand{\CSLBlock}[1]{#1\hfill\break}
\newcommand{\CSLLeftMargin}[1]{\parbox[t]{\csllabelwidth}{#1}}
\newcommand{\CSLRightInline}[1]{\parbox[t]{\linewidth - \csllabelwidth}{#1}\break}
\newcommand{\CSLIndent}[1]{\hspace{\cslhangindent}#1}

\begin{document}


\title{Enhancing statistical reproducibility in the biomedical sciences:
practical guides for the everyday researcher }


    \titlerunning{Enhancing statistical reproducibility in the
biomedical sciences}

\author{  Ariel Mundo Ortiz \and  Bouchra Nasri \and  }


\institute{
        Ariel Mundo Ortiz \at
     École de santé publique, Université de Montréal, \newline Centre de
recherches mathématiques, Université de Montréal \\
     \email{\href{mailto:ariel.mundo.ortiz@umontreal.ca}{\nolinkurl{ariel.mundo.ortiz@umontreal.ca}}}  %  \\
%             \emph{Present address:} of F. Author  %  if needed
    \and
        Bouchra Nasri \at
     École de santé publique, Université de Montréal \\
     \email{\href{mailto:bouchra.nasri@umontreal.ca}{\nolinkurl{bouchra.nasri@umontreal.ca}}}  %  \\
%             \emph{Present address:} of F. Author  %  if needed
    \and
    }

\date{Received: date / Accepted: date}
% The correct dates will be entered by the editor


\maketitle

\begin{abstract}
Reproducibility continues to be a major goal of biomedical research.
However, the field still struggles to use Statistics (a core component
of reproducibility) in a reproducible way. In this paper, we present
some of the reasons that we believe contribute to this ongoing
disconnect between Statistics and reproducible biomedical research, and
we also present guidelines aimed to help trainees and researchers to
internalize the need of using Statistics in a manner that enhances
reproducibility.
\\
\keywords{
        keywordA, keywordB \and
    }


\end{abstract}


\def\spacingset#1{\renewcommand{\baselinestretch}%
{#1}\small\normalsize} \spacingset{1}


\hypertarget{background}{%
\section{Background}\label{background}}

The ``reproducibility crisis'' in science has affected particularly
biomedical research , where it is believed that many articles describe
results that cannot be replicated\textsuperscript{1,2}. Although it is
recognized that some indicators of reproducibility have improved in
recent years in publications in the field, it is also recognized that
there is still a large room for improvement\textsuperscript{3}.

In recent years, different studies have identified practices that could
be implemented to improve various aspects of reproducibility in the
field, suggesting (among other recommendations) greater involvement of
senior investigators in the data collection/analysis
process\textsuperscript{4}, and the use of reproducible tools and
workflows\textsuperscript{5,6}. However, the incorrect use of statistics
is also a key factor that limits reproducibility\textsuperscript{7,8}.

It is well established that statistics are misused in biomedical
research\textsuperscript{8}, but we believe it is important to also
highlight some of the causes that drive this misuse in order if we are
to improve and remedy this issue. We believe that an important factor
that drives the improper use of statistics is the fact that statistics
is a subject that is seen as abstract and obscure by biomedical trainees
and researchers. In fact, it is known that statistics is something that
worries students\textsuperscript{9}. The biomedical field is not exempt
from seeing statistics as a ``necessary evil'' and consequently,
continuing to perpetuate systemic issues that affect reproducibility.
Overall, this results in a disconnect between what \emph{should} be done
to address statistical reproducibility in biomedical research, and the
approach that \emph{is} taken by the field.

We believe that presenting guidelines that address some of the issues
that the field faces in terms of statistical reproducibility is a
necessary step towards improving statistical practices and preparing
future generations of scientists to routinely perform research that is
open and reproducible.

\hypertarget{guidelines}{%
\section{Guidelines}\label{guidelines}}

\hypertarget{why-are-we-doing-this}{%
\subsection{``Why are we doing this?''}\label{why-are-we-doing-this}}

The very abstract and quantitative nature of Statistics often leads
students to worry and to adopt a ``memorisation'' strategy to approach
it\textsuperscript{10}, a phenomenon that trainees in biomedical
research are not exempt of. However, memory alone does not suffice to
successfully analyze ``real'' data: it has been shown that researchers
often incorrectly analyze experimental designs, and checks that are to
be made before experiments are done after experiments have been
completed\textsuperscript{11}.

It is therefore important that trainees focus on understanding the
``why'' (the rationale) of the statistical tools they intend to use.
Indeed, Statistics cannot be appropriately used without understanding
the assumptions and the basic theory that underlies
them\textsuperscript{12}. Researchers tend to think that Statistics are
only needed when a statistical test needs to be performed, but the truth
of the matter is that Statistics play a crucial role at every stage of
research: they are needed to determine the number of observations
required to achieve a certain statistical power, as well as to determine
the choice of experimental design that needs to be used to answer the
question of interest. These are steps that precede a ``statistical
test'', and they need to be taken into careful consideration before any
experiments or data are to be collected.

\hypertarget{learning-statistics-is-necessary}{%
\subsection{Learning statistics is
necessary}\label{learning-statistics-is-necessary}}

In the previous point we have indicated that researchers need to learn
the foundations of Statistics in order to adequately use them. This is a
view shared by many others, that have emphasized the importance of
statistical education as a way of addressing the reproducibility
crisis\textsuperscript{13}. However, the very nature of Statistics makes
it distinct from learning physics, or mathematics\textsuperscript{10}.
It has been suggested that a problem-based learning approach might be
best suited to teach Statistics\textsuperscript{14}, but it is possible
that most trainees will not be able to benefit from that approach in the
Statistics classes they are required to take (if that is the case). This
implies that the most likely option for a trainee is a combination of
formal Statistical training (taking classes in a University) and
self-guided study. After all, not only there is a wide range of
variation among the academic requirements set by each
program\textsuperscript{15} but it is also possible that trainees are
interested in a particular statistical topic without committing the time
and resources needed for a full course.

Fortunately, the educational resources of Statistics have increased
vastly over the last decade, and nowadays there are multiple materials
that cover a wide range of statistical topics without excessive
mathematical complexity. For example, those interested in revisiting
statistical foundations will find an excellent resource in the work of
James et al.\textsuperscript{16}. Topics on generalized linear models
and generalized linear mixed models (which we believe are extremely
important for biomedical researchers to learn, but that might not be
covered in the Statistics courses required by their programs) can be
found in McCulloch\textsuperscript{17}, Dobson\textsuperscript{18}, and
Stroup\textsuperscript{19}.

\hypertarget{you-should-not-aim-to-do-everybody-does}{%
\subsection{You should not aim to do ``everybody
does''}\label{you-should-not-aim-to-do-everybody-does}}

This point might seem redundant in the light of what we indicated above
about the importance of learning Statistics. However, our own experience
has shown that a deterrent for trainees to learn Statistics is that it
suffices to mimic the analyses they have seen in a paper (``what
everybody does''). It is tempting to repeat the analysis presented in a
previous study, as it might seem that because the study passed peer
review, its methodology (including its statistical analyses)
\emph{should} be correct. However, without a clear understanding of the
assumptions behind the analysis is correct in the first place. The truth
is that that is seldom the case, as Hardwicke et al.\textsuperscript{20}
showed that most leading biomedical journals do not perform specialized
statistical reviews on papers they published (only 23\% reported that
they did statistical analyses).

Sadly, statistical errors plague biomedical studies\textsuperscript{21},
and without a solid understanding of the assumptions and limitations of
a statistical method, researchers will be unable to critically assess
the analyses presented and determine if the methodology presented is
applicable in their case. Peer review is not perfect, and researchers
need to internalize the fact that repetition is not enough to achieve
reproducibility.

\hypertarget{your-statistical-analyses-need-to-be-reproducible}{%
\subsection{Your statistical analyses need to be
reproducible}\label{your-statistical-analyses-need-to-be-reproducible}}

Biomedical data is complex and nowadays, computational skills are
necessary to successfully analyze the datasets that are obtained as a
result of experiments\textsuperscript{5}. However, there seems to be gap
between the computational skills that trainees acquire to collect their
data, and the skills they possess to perform statistical analyses of
such data. Because of the ``memorizing'' approach that we have mentioned
above, many researchers prefer a ``click'' approach to perform their
statistical analyses (using a program that only requires them to select
certain options from a menu). Although a ``click'' approach is
apparently more efficient from a time perspective, the biggest trade-off
is that there is no real understanding from the user of what is actually
happening behind the scenes\textsuperscript{22}.

Others have indicated the importance of coding to create a workflow that
minimizes the errors associated with manual
manipulation\textsuperscript{23}, an opinion we concur with and believe
is equally applicable in the context of statistical analyses. Although
statistical analyses can be performed by a myriad of different
computational tools (such as R, Python, and Julia), no statistical
analysis is complete if it is not reproducible.

There are some tools that are specifically suited to allow reproducible
workflows. Following on the recommendations of Brito et
al.\textsuperscript{5} regarding the use of open source tools to create
reproducible workflows, in Table 1 we provide a list of open source
tools that are designed to combine text and computations (therefore
allowing to create reproducible documents), that are easily accessible
to biomedical researchers, that support multiple computational
languages, and that have multiple resources (such as examples, books,
and guides) that can help researchers familiarize themselves with how
they work.

\footnotesize

\begin{longtable}[]{@{}
  >{\raggedright\arraybackslash}p{(\columnwidth - 6\tabcolsep) * \real{0.1587}}
  >{\centering\arraybackslash}p{(\columnwidth - 6\tabcolsep) * \real{0.2857}}
  >{\centering\arraybackslash}p{(\columnwidth - 6\tabcolsep) * \real{0.3175}}
  >{\centering\arraybackslash}p{(\columnwidth - 6\tabcolsep) * \real{0.2381}}@{}}
\caption{Tools that allow for reproducible statistical
analyses}\tabularnewline
\toprule()
\begin{minipage}[b]{\linewidth}\raggedright
Tool
\end{minipage} & \begin{minipage}[b]{\linewidth}\centering
Characteristics
\end{minipage} & \begin{minipage}[b]{\linewidth}\centering
Languages supported
\end{minipage} & \begin{minipage}[b]{\linewidth}\centering
Resources
\end{minipage} \\
\midrule()
\endfirsthead
\toprule()
\begin{minipage}[b]{\linewidth}\raggedright
Tool
\end{minipage} & \begin{minipage}[b]{\linewidth}\centering
Characteristics
\end{minipage} & \begin{minipage}[b]{\linewidth}\centering
Languages supported
\end{minipage} & \begin{minipage}[b]{\linewidth}\centering
Resources
\end{minipage} \\
\midrule()
\endhead
RMarkdown & Allows to create reproducible documents (notebooks, reports,
books, scientific articles) that combine coding and text. Output formats
include HTML, PDF, MS Word, Beamer and others & R, Python, SQL, Julia
and others & Xie et al.\textsuperscript{24}, an online version of the
book can be found at \url{https://bookdown.org/yihui/rmarkdown/}),
examples can be found at
\url{https://rmarkdown.rstudio.com/gallery.html} \\
Bookdown & Allows to create reproducible documents and follows the same
syntax of RMarkdown, but includes added capabilities such as
cross-referencing and facilitating the creation of documents (such as
books) that are composed of multiple RMarkdown documents. & R, C/C++,
Python, Fortran, Julia, SQL, Stan and others & Xie\textsuperscript{25},
an online version of the book can be found at
\url{https://bookdown.org/yihui/bookdown/} \url{https://bookdown.org/}
contains examples of books created using Bookdown. \\
Quarto & Publishing system for scientific and technical documents that
is compatible with VS Code, RStudio, and Jupyter Notebooks. Documents
can be compiled in HTML, PDF, MS Word, Beamer, Shiny, MS PowerPoint,
Revealjs presentations, and many others & R, Python, Julia, Observable &
\url{https://quarto.org/} contains multiple examples, tutorials, and use
guides \\
\bottomrule()
\end{longtable}

\normalsize

\hypertarget{models-are-just-that-models}{%
\subsection{Models are just that,
models}\label{models-are-just-that-models}}

\begin{itemize}
\item
  Biology is complex
\item
  Models are a simplification
\item
  They offer an explanation, but that is not the only explanation
\end{itemize}

\hypertarget{significance-should-not-be-driven-by-a-p-value}{%
\subsection{Significance should not be driven by a
p-value}\label{significance-should-not-be-driven-by-a-p-value}}

Perhaps the aspect of statistical reproducibility that biomedical
researchers struggle the most is the concept of significance and its
association with a \emph{p-value} below 0.05. Much has been said about
the limitations of statistical tests and p-values, and how it is wrong
to dichotomize the ``significance'' of a result on the basis of a
p-value cutoff, and yet, this is still a prevalent practice in the
field.

Here, we will not attempt to provide another repetition of the facts
that others have so eloquently provided about this topic (we refer the
reader to the excellent works of Ziliak and
McCloskey\textsuperscript{26}, Greenland et al.\textsuperscript{27},
Wasserstein and Lazar\textsuperscript{28}, and Chia\textsuperscript{29}
for discussions in detail), but we much rather try to shed some light on
why the ``\emph{p}-values \textless0.05 equals significance'' is so
prevalent in biomedical research.

We believe that this problem has multiple facets: First, there is the
issue of how researchers view statistics, which closely relates to that
we described in Point 1. Researchers view Statistics as a black box
where multiple obscure terms such as ``distribution'', ``likelihood'',
``parameters'', and many greek letters are mixed up along a language
whose technicalities are incomprehensible and confusing. In a sense, it
is true that technical statistical language is a completely different
beast from the research language that biomedical researchers commonly
employ; comprehensibly, trying to learn a new technical language might
seem as a daunting task for which researchers, facing already time
constraints due to research and academic life, might not feel to have
the time or resources to learn. Adding to this problematic, introductory
statistical courses and textbooks typically do not discuss the
limitations of \emph{p}-values in a clear way\textsuperscript{27}.

Second, the dichotomization of significance is the driving force that
the field uses to measure research outcomes. In other words, researchers
perpetually suffer from ``significant-itis''\textsuperscript{29}
(believing that results are only good if the p-value is \textless0.05)
because such metrics are ubiquitously presented in publications as the
correct metric to measure the success or failure of a research outcome.

These two facts then, create a vicious cycle where researchers are
perpetually working to find ``significant results'', and makes trainees
believe early on that that is the correct way of validating research
outcomes. We believe it is important to remind researchers that
statistical significance does not equate to clinical
significance\textsuperscript{27}, that \emph{p}-values are just a way to
determine how the data behaves under the model assumptions, and that
they are the result of historical and philosophical choices made by
people\textsuperscript{30,31}.

The other part of the problem is what the view of statistical tests as
an obscure and complicat

p-values should not be the final goal of some data, but rather the
interpretation by the researcher is what is important.

\begin{itemize}
\item
  p-values and positive results continue to dominate the field
\item
  what a p-value actually tells (maybe example with data)?
\item
  p-value should not be the goal of a study
\end{itemize}

\hypertarget{references}{%
\section*{References}\label{references}}
\addcontentsline{toc}{section}{References}

\hypertarget{refs}{}
\begin{CSLReferences}{0}{0}
\leavevmode\vadjust pre{\hypertarget{ref-begley2015}{}}%
\CSLLeftMargin{1. }%
\CSLRightInline{Begley CG, Ioannidis JP. Reproducibility in science:
Improving the standard for basic and preclinical research.
\emph{Circulation research}. 2015;116(1):116-126.}

\leavevmode\vadjust pre{\hypertarget{ref-oakdenrayner2018}{}}%
\CSLLeftMargin{2. }%
\CSLRightInline{Oakden-Rayner L, Beam AL, Palmer LJ. Medical journals
should embrace preprints to address the reproducibility crisis.
\emph{International Journal of Epidemiology}. 2018;47(5):1363-1365.
doi:\href{https://doi.org/10.1093/ije/dyy105}{10.1093/ije/dyy105}}

\leavevmode\vadjust pre{\hypertarget{ref-wallach2018}{}}%
\CSLLeftMargin{3. }%
\CSLRightInline{Wallach JD, Boyack KW, Ioannidis JP. Reproducible
research practices, transparency, and open access data in the biomedical
literature, 2015--2017. \emph{PLoS biology}. 2018;16(11):e2006930.}

\leavevmode\vadjust pre{\hypertarget{ref-samsa2019}{}}%
\CSLLeftMargin{4. }%
\CSLRightInline{Samsa G, Samsa L. A guide to reproducibility in
preclinical research. \emph{Academic Medicine}. 2019;94(1):47-52.
doi:\href{https://doi.org/10.1097/acm.0000000000002351}{10.1097/acm.0000000000002351}}

\leavevmode\vadjust pre{\hypertarget{ref-brito2020}{}}%
\CSLLeftMargin{5. }%
\CSLRightInline{Brito JJ, Li J, Moore JH, et al. {Recommendations to
enhance rigor and reproducibility in biomedical research}.
\emph{GigaScience}. 2020;9(6).
doi:\href{https://doi.org/10.1093/gigascience/giaa056}{10.1093/gigascience/giaa056}}

\leavevmode\vadjust pre{\hypertarget{ref-papin2020}{}}%
\CSLLeftMargin{6. }%
\CSLRightInline{Papin JA, Gabhann FM, Sauro HM, Nickerson D, Rampadarath
A. Improving reproducibility in computational biology research.
\emph{{PLOS} Computational Biology}. 2020;16(5):e1007881.
doi:\href{https://doi.org/10.1371/journal.pcbi.1007881}{10.1371/journal.pcbi.1007881}}

\leavevmode\vadjust pre{\hypertarget{ref-montgomery2019}{}}%
\CSLLeftMargin{7. }%
\CSLRightInline{Erwin B. Montgomery Jr. \emph{Reproducibility in
Biomedical Research}. Elsevier; 2019.
doi:\href{https://doi.org/10.1016/c2018-0-02296-3}{10.1016/c2018-0-02296-3}}

\leavevmode\vadjust pre{\hypertarget{ref-thiese2015}{}}%
\CSLLeftMargin{8. }%
\CSLRightInline{Thiese MS, Arnold ZC, Walker SD. The misuse and abuse of
statistics in biomedical research. \emph{Biochemia medica}.
2015;25(1):5-11.}

\leavevmode\vadjust pre{\hypertarget{ref-ralston2019}{}}%
\CSLLeftMargin{9. }%
\CSLRightInline{Ralston K. {``Sociologists shouldn't have to study
statistics''}: Epistemology and anxiety of statistics in sociology
students. \emph{Sociological Research Online}. 2019;25(2):219-235.
doi:\href{https://doi.org/10.1177/1360780419888927}{10.1177/1360780419888927}}

\leavevmode\vadjust pre{\hypertarget{ref-ramsey1999}{}}%
\CSLLeftMargin{10. }%
\CSLRightInline{Ramsey JB. Why do students find statistics so difficult.
\emph{Proccedings of the 52th Session of the ISI Helsinki}. Published
online 1999:10-18.}

\leavevmode\vadjust pre{\hypertarget{ref-kitchenham2019}{}}%
\CSLLeftMargin{11. }%
\CSLRightInline{Kitchenham B, Madeyski L, Brereton P. Problems with
statistical practice in human-centric software engineering experiments.
In: \emph{Proceedings of the Evaluation and Assessment on Software
Engineering}. {ACM}; 2019.
doi:\href{https://doi.org/10.1145/3319008.3319009}{10.1145/3319008.3319009}}

\leavevmode\vadjust pre{\hypertarget{ref-marino2018}{}}%
\CSLLeftMargin{12. }%
\CSLRightInline{Marino MJ. Statistical analysis in preclinical
biomedical research. In: \emph{Research in the Biomedical Sciences}.
Elsevier; 2018:107-144.
doi:\href{https://doi.org/10.1016/b978-0-12-804725-5.00003-3}{10.1016/b978-0-12-804725-5.00003-3}}

\leavevmode\vadjust pre{\hypertarget{ref-patil2022}{}}%
\CSLLeftMargin{13. }%
\CSLRightInline{Patil S, Satagopan J. Building and teaching a statistics
curriculum for post-doctoral biomedical scientists at a free-standing
cancer center. \emph{{CHANCE}}. 2022;35(1):56-64.
doi:\href{https://doi.org/10.1080/09332480.2022.2039036}{10.1080/09332480.2022.2039036}}

\leavevmode\vadjust pre{\hypertarget{ref-ekmekci2012}{}}%
\CSLLeftMargin{14. }%
\CSLRightInline{Ekmekci O, Hancock AB, Swayze S. Teaching statistical
research methods to graduate students: Lessons learned from three
different degree programs. \emph{International Journal of Teaching and
Learning in Higher Education}. 2012;24(2):272-279.}

\leavevmode\vadjust pre{\hypertarget{ref-gatchell2014}{}}%
\CSLLeftMargin{15. }%
\CSLRightInline{Gatchell D, Linsenmeier R. Similarities and differences
in undergraduate biomedical engineering curricula in the united states.
In: \emph{2014 {ASEE} Annual Conference {\&} Exposition Proceedings}.
{ASEE} Conferences.
doi:\href{https://doi.org/10.18260/1-2--23015}{10.18260/1-2-\/-23015}}

\leavevmode\vadjust pre{\hypertarget{ref-james2021}{}}%
\CSLLeftMargin{16. }%
\CSLRightInline{James G, Witten D, Hastie T, Tibshirani R. \emph{An
Introduction to Statistical Learning}. Springer {US}; 2021.
doi:\href{https://doi.org/10.1007/978-1-0716-1418-1}{10.1007/978-1-0716-1418-1}}

\leavevmode\vadjust pre{\hypertarget{ref-mcculloch2004}{}}%
\CSLLeftMargin{17. }%
\CSLRightInline{McCulloch CE, Searle SR. \emph{Generalized, Linear, and
Mixed Models}. John Wiley \& Sons; 2004.}

\leavevmode\vadjust pre{\hypertarget{ref-dobson2018}{}}%
\CSLLeftMargin{18. }%
\CSLRightInline{Dobson AJ, Barnett AG. \emph{An Introduction to
Generalized Linear Models, Fourth Edition}. 4th ed. CRC Press; 2018.}

\leavevmode\vadjust pre{\hypertarget{ref-stroup2012}{}}%
\CSLLeftMargin{19. }%
\CSLRightInline{Stroup WW. \emph{Generalized Linear Mixed Models}. CRC
Press; 2012.}

\leavevmode\vadjust pre{\hypertarget{ref-hardwicke2020}{}}%
\CSLLeftMargin{20. }%
\CSLRightInline{Hardwicke TE, Goodman SN. How often do leading
biomedical journals use statistical experts to evaluate statistical
methods? The results of a survey. Koletsi D, ed. \emph{{PLOS} {ONE}}.
2020;15(10):e0239598.
doi:\href{https://doi.org/10.1371/journal.pone.0239598}{10.1371/journal.pone.0239598}}

\leavevmode\vadjust pre{\hypertarget{ref-lang2004}{}}%
\CSLLeftMargin{21. }%
\CSLRightInline{Lang T. Twenty statistical errors even you can find in
biomedical research articles. \emph{Croatian Medical Journal}.
2004;45:361-370.}

\leavevmode\vadjust pre{\hypertarget{ref-deardorff2020}{}}%
\CSLLeftMargin{22. }%
\CSLRightInline{Deardorff A. Why do biomedical researchers learn to
program? An exploratory investigation. \emph{Journal of the Medical
Library Association: JMLA}. 2020;108(1):29.}

\leavevmode\vadjust pre{\hypertarget{ref-sandve2013}{}}%
\CSLLeftMargin{23. }%
\CSLRightInline{Sandve GK, Nekrutenko A, Taylor J, Hovig E. Ten simple
rules for reproducible computational research. Bourne PE, ed.
\emph{{PLoS} Computational Biology}. 2013;9(10):e1003285.
doi:\href{https://doi.org/10.1371/journal.pcbi.1003285}{10.1371/journal.pcbi.1003285}}

\leavevmode\vadjust pre{\hypertarget{ref-xie2018}{}}%
\CSLLeftMargin{24. }%
\CSLRightInline{Xie Y, Allaire JJ, Grolemund G. \emph{{R} Markdown}. CRC
Press; 2018.}

\leavevmode\vadjust pre{\hypertarget{ref-xie2016}{}}%
\CSLLeftMargin{25. }%
\CSLRightInline{Xie Y. Bookdown: Authoring books and technical documents
with r markdown. In: \emph{The {R} Series}. CRC Press; 2016:47-66.}

\leavevmode\vadjust pre{\hypertarget{ref-ziliak2008}{}}%
\CSLLeftMargin{26. }%
\CSLRightInline{Ziliak ST, McCloskey D. \emph{The Cult of Statistical
Significance: How the Standard Error Costs Us Jobs, Justice, and Lives}.
University of Michigan Press; 2008.}

\leavevmode\vadjust pre{\hypertarget{ref-greenland2016}{}}%
\CSLLeftMargin{27. }%
\CSLRightInline{Greenland S, Senn SJ, Rothman KJ, et al. Statistical
tests, p values, confidence intervals, and power: A guide to
misinterpretations. \emph{European Journal of Epidemiology}.
2016;31(4):337-350.
doi:\href{https://doi.org/10.1007/s10654-016-0149-3}{10.1007/s10654-016-0149-3}}

\leavevmode\vadjust pre{\hypertarget{ref-wasserstein2016}{}}%
\CSLLeftMargin{28. }%
\CSLRightInline{Wasserstein RL, Lazar NA. The {ASA} statement on
\emph{p}-values: Context, process, and purpose. \emph{The American
Statistician}. 2016;70(2):129-133.
doi:\href{https://doi.org/10.1080/00031305.2016.1154108}{10.1080/00031305.2016.1154108}}

\leavevmode\vadjust pre{\hypertarget{ref-chia1997}{}}%
\CSLLeftMargin{29. }%
\CSLRightInline{Chia K-S. "Significant-itis" --- an obsession with the
p-value. \emph{Scandinavian Journal of Work, Environment \& Health}.
1997;23(2):152-154. Accessed December 22, 2022.
\url{http://www.jstor.org/stable/40966624}}

\leavevmode\vadjust pre{\hypertarget{ref-huberty1993}{}}%
\CSLLeftMargin{30. }%
\CSLRightInline{Huberty CJ. Historical origins of statistical testing
practices. \emph{The Journal of Experimental Education}.
1993;61(4):317-333.
doi:\href{https://doi.org/10.1080/00220973.1993.10806593}{10.1080/00220973.1993.10806593}}

\leavevmode\vadjust pre{\hypertarget{ref-freedman1999}{}}%
\CSLLeftMargin{31. }%
\CSLRightInline{Freedman D. From association to causation : Some remarks
on the history of statistics. \emph{Journal de la Société française de
statistique}. 1999;140(3):5-32.
\url{http://www.numdam.org/item/JSFS_1999__140_3_5_0/}}

\end{CSLReferences}


\bibliographystyle{spbasic}
\bibliography{bibliography/refs.bib}


\end{document}
